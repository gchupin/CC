\documentclass[10pt, a4paper]{article}

\usepackage[T1]{fontenc}
\usepackage[utf8]{inputenc}
\usepackage[francais]{babel}

\title{DevoirMaison1}
\author {CHUPIN Guillaume et PILLEUX Julien}

\begin{document}
\maketitle
\thispagestyle {empty}
\newpage
\tableofcontents
\newpage

\section {HAC est NP-difficile}
\begin {itemize}
\item Une réduction polynomiale du problème chemin hamiltonien vers le problème HAC serait de prendre le début de notre chemin hamiltonien comme la racine de l'arbre et comme fils le suivant dans le chemin et ainsi de suite (on obtiendra ainsi un arbre couvrant en forme de ligne et de hauteur la taille du chemin hamiltonien).
\item Le problème chemin hamiltonien étant NP-complet on en conclut que HAC est NP-difficile (car les réductions polynomiale sont transitive et que tout problèmes NP peut se réduire polynomialement en chemin hamiltonien, car il est NP complet)
\end {itemize}
\section {Réduction de HAC vers SAT}
\begin {enumerate}
\item Pour chaque sommet v $\in$ V, il y a au moins un entier h tq $x_{v,h}$ est vrai(Chaque sommet a au moins une hauteur dans l'arbre):
  \[
  \bigwedge_{v \in V}\bigvee_{h=0}^nx_{v,h}
  \]
\item  Pour chaque sommet v $\in$ V, il y a au plus un entier h tq $x_{v,h}$ est vrai(Chaque sommet n'a qu'une hauteur dans l'arbre).
  \[
  \bigwedge_{v\in V}\bigwedge_{h\neq k}(\neg x_{v,h} \vee \neg x_{v,k})
  \]
\item Il y a un unique sommet v tq d(v) = 0 (``v est la racine''):
  \[
  \bigwedge_{v\neq w}(\neg x_{v,0} \vee \neg x_{w,0})
  \]
\item Il y a au moins un sommet v tq d(v) = k(Il y a au moins un sommet par hauteur):
  \[
  \bigwedge_{k=0}^{n}\bigvee_{v\in V}x_{v,k}
  \]
\item Pour chaque sommet v, si d(v) > 0, alors il existe un sommet u tel que uv $\in$ E et d(u) = d(v) - 1 (``le sommet u est un parent potentiel de v dans l'arbre'')(donc si u est parent de v dans l'arbre alors u et v sont voisins dans le graphe):
  \[
  \bigwedge_{v\in V}\bigwedge_{h=1}^n\bigvee_{uv\in E}(\neg x_{v,h}\vee x_{u,h-1})
  \]
\end {enumerate}
\newpage
Toute ces clauses nous garantissent d'avoir la réduction de sat vers HAC, car on a bien une seul racine, puis il y a au moins 1 sommet sur toute les profondeurs
recherché, ensuite chaque sommet du graphe est dans l'arbre et chaque arrête de l'arbre correspond bien a une arrête du graphe.\newline
De plus, la réduction est polynomiale, car chaque clause est bornée par un polynôme.\newline
En effet:
\begin {itemize}
\item La première clause est bornée par le nombre de sommet.
\item La seconde clause est bornée par le nombre de sommet * $n^2$(n étant la profondeur de l'arbre couvrant) qui est polynomiale si n n'est pas trop grand(et n ne peut pas être supérieur au nombre de sommet sinon il ne peut assurément pas y avoir d'arbre couvrant de hauteur n).
\item La troisième clause est bornée par le nombre de sommet au carré.
\item La quatrième clause est bornée par n.
\item La cinquième clause est bornée par le nombre de sommet * n
\end {itemize}

\section {Optimisations}
Une optimisation utilisé pour la 2ème clause, est de prendre k~>~h (ainsi on ne traite pas deux fois le même couple de littéraux). On peut aussi réduire le nombre de littéraux de la 5ème clause en n'affichant qu'une fois le littéral $\neg x_{v,h}$.\newline
Une optimisation utilisé dans le code, est de faire un parcours en profondeur, pour tester si le graphe est connexe et si il ne l'est pas, on ne fait pas la réduction, car il ne peut pas y avoir d'arbre couvrant dans un graphe qui n'est pas connexe.
On teste aussi que la profondeur, rentré par l'utilisateur, soit inférieur au nombre de sommet du graphe (car un graphe ne peut pas avoir un arbre couvrant de profondeur supérieur à son nombre de sommet).


\end{document}
